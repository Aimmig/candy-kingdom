\documentclass{llncs}

\pagestyle{headings}

\usepackage{graphicx}
\usepackage{amsmath}
\usepackage{amssymb}
\usepackage[ruled,linesnumbered,vlined]{algorithm2e}
\usepackage{color}
\usepackage{tabularx}
\usepackage{todonotes}
\usepackage{tabto}
\usepackage{url}
\usepackage{csquotes}
\usepackage[T1]{fontenc} 
\usepackage[utf8]{inputenc}

\usepackage{tikz}
\usetikzlibrary{shapes,arrows,chains,matrix,fit,backgrounds,calc}

\usepackage{mathtools}
\DeclarePairedDelimiter\ceil{\lceil}{\rceil}
\DeclarePairedDelimiter\floor{\lfloor}{\rfloor}

\newcommand{\assign}{\ensuremath{\leftarrow }}

\newcommand{\underapp}[1]{\ensuremath{#1^\downarrow}}
\newcommand{\overapp}[1]{\ensuremath{#1^\uparrow}}

\newcommand{\True}{\ensuremath{\top}}
\newcommand{\False}{\ensuremath{\bot}}
\newcommand{\Undef}{\ensuremath{\mathbf{U}}}
\newcommand{\SAT}{\ensuremath{\mathsf{SAT}}}
\newcommand{\UNSAT}{\ensuremath{\mathsf{UNSAT}}}
\newcommand{\UNDEF}{\ensuremath{\mathsf{UNDEF}}}

\newcommand{\Vars}[1]{\mathsf{vars}(#1)}
\newcommand{\Lits}[1]{\mathsf{lits}(#1)}
\newcommand{\Enc}[1]{\mathsf{enc}(#1)}

\makeatletter
\def\blfootnote{\gdef\@thefnmark{}\@footnotetext}
\makeatother

\title{System Description of the Candy Family of SAT Solvers}
\author{Markus Iser, Felix Kutzner}
\institute{
  Karlsruhe Institute of Technology (KIT), Germany\\
  \url{{markus.iser, felix.kutzner}@kit.edu}
}

\begin{document}

\maketitle

\begin{abstract}

\end{abstract}

\begin{itemize}
\item Glucose Branch (STL Datastructures)
\item ClauseAllocator (Size Buckets, Revamping, Small Clause-Headers, Clauses are Pointers, Size Buckets)
\item Watcher sorting
\item Clauses are pointers
\item Inprocessing: Simplify, Subsumption, Self-Subsuming Resolution (no eager simplify)
\item Optimizations and Certificate also in Incremental Mode 
\item Gate Analysis and Random Simulation (RSAR and RSIL Variants of Candy Solver)
\item improved extensibility via templates
\item modular toolbox approach: gate recognition, RSIL, RSAR, IPASIR
\item miter detection (RSILmd)
\item code quality: > 200 tests, continuous integration
\end{itemize}  
    

\bibliographystyle{splncs03}
%\bibliographystyle{abbrv}
\bibliography{abstractionrefinement,blockedsets,general,preprocessing,structurerecognition,applications,encodings,otherstructure,self}

\end{document}

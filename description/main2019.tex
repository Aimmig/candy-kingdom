\documentclass[conference]{IEEEtran}

\pagestyle{headings}

\usepackage{graphicx}
\usepackage{amsmath}
\usepackage{amssymb}
\usepackage[ruled,linesnumbered,vlined]{algorithm2e}
\usepackage{color}
\usepackage{tabularx}
\usepackage{todonotes}
\usepackage{tabto}
\usepackage{url}
\usepackage{csquotes}
\usepackage[T1]{fontenc} 
\usepackage[utf8]{inputenc}

\usepackage{tikz}
\usetikzlibrary{shapes,arrows,chains,matrix,fit,backgrounds,calc}

\usepackage{mathtools}
\DeclarePairedDelimiter\ceil{\lceil}{\rceil}
\DeclarePairedDelimiter\floor{\lfloor}{\rfloor}

\usepackage{cite}

\newcommand{\assign}{\ensuremath{\leftarrow }}

\newcommand{\underapp}[1]{\ensuremath{#1^\downarrow}}
\newcommand{\overapp}[1]{\ensuremath{#1^\uparrow}}

\newcommand{\True}{\ensuremath{\top}}
\newcommand{\False}{\ensuremath{\bot}}
\newcommand{\Undef}{\ensuremath{\mathbf{U}}}
\newcommand{\SAT}{\ensuremath{\mathsf{SAT}}}
\newcommand{\UNSAT}{\ensuremath{\mathsf{UNSAT}}}
\newcommand{\UNDEF}{\ensuremath{\mathsf{UNDEF}}}

\newcommand{\Vars}[1]{\mathsf{vars}(#1)}
\newcommand{\Lits}[1]{\mathsf{lits}(#1)}
\newcommand{\Enc}[1]{\mathsf{enc}(#1)}

\makeatletter
\def\blfootnote{\gdef\@thefnmark{}\@footnotetext}
\makeatother

\title{Candy for SAT Race 2019}
\author{\IEEEauthorblockN{Markus Iser\IEEEauthorrefmark{1}
and Felix Kutzner\IEEEauthorrefmark{2}}
\IEEEauthorblockA{Karlsruhe Institute of Technology (KIT)\\
Karlsruhe, Germany\\
\IEEEauthorrefmark{1}markus.iser@kit.edu,
\IEEEauthorrefmark{2}felix@kutzner.io}}

\begin{document}

\maketitle

\begin{abstract}
We use Candy as a platform to systematically analyse the properties of competing strategies in a portfolio. 
\end{abstract}


\section{Summary}
\textbf{Candy}~\cite{CandyGithub} is a fork of \textbf{Glucose 3} \cite{Audemard:2009:Glucose,Niklas:2003:Minisat}.
Candy provides a flexible and efficient architecture to experiment with many different strategies. 
The solver does this by orchestrating a set of loosely coupled systems, 
which mainly provide an interface for implementations of competing strategies. 
For example, Candy provides a variety of strategies in the branching-system. 


\section{Implementation}
Among others, the branching system can resort to implementations of gate-analysis and random-simulation based \emph{implicit learning} (RSIL) which uses the algorithms which we presented in~\cite{Iser:2017:RandomSimulation} and~\cite{Iser:2015:GateRecognition}. 

Now, there exists also a parallel mode with selected combinations and configurations of strategies and efficient memory sharing.


\section{Interfaces}
Candy provides an IPASIR interface~\cite{IpasirGithub} and an interface to the generic massively parallel SAT solver HordeSAT~\cite{HordesatGithub}. 
The sonification interface makes solver runs even audible~\cite{CandySonification}. 


\section{Candy in SAT Race 2019}
We submitted Candy in its default setting which is a configuration of strategies that is roughly similar to the one used in Glucose 3. 
This is the public evaluation of our baseline performance for reference. 


\bibliographystyle{splncs03}
%\bibliographystyle{abbrv}
\IEEEtriggeratref{3}
\bibliography{main}

\end{document}

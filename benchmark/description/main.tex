\documentclass[conference]{IEEEtran}

\pagestyle{headings}

\usepackage{graphicx}
\usepackage{amsmath}
\usepackage{amssymb}
%\usepackage[ruled,linesnumbered,vlined]{algorithm2e}
\usepackage{color}
\usepackage{tabularx}
\usepackage{todonotes}
\usepackage{tabto}
\usepackage{hyperref}
\usepackage{csquotes}
\usepackage[T1]{fontenc} 
\usepackage[utf8]{inputenc}
\usepackage{listings}

\usepackage{tikz}
\usetikzlibrary{shapes,arrows,chains,matrix,fit,backgrounds,calc}

\usepackage{mathtools}
\DeclarePairedDelimiter\ceil{\lceil}{\rceil}
\DeclarePairedDelimiter\floor{\lfloor}{\rfloor}

\newcommand{\assign}{\ensuremath{\leftarrow }}

\newcommand{\underapp}[1]{\ensuremath{#1^\downarrow}}
\newcommand{\overapp}[1]{\ensuremath{#1^\uparrow}}

\newcommand{\True}{\ensuremath{\top}}
\newcommand{\False}{\ensuremath{\bot}}
\newcommand{\Undef}{\ensuremath{\mathbf{U}}}
\newcommand{\SAT}{\ensuremath{\mathsf{SAT}}}
\newcommand{\UNSAT}{\ensuremath{\mathsf{UNSAT}}}
\newcommand{\UNDEF}{\ensuremath{\mathsf{UNDEF}}}

\newcommand{\Vars}[1]{\mathsf{vars}(#1)}
\newcommand{\Lits}[1]{\mathsf{lits}(#1)}
\newcommand{\Enc}[1]{\mathsf{enc}(#1)}

\makeatletter
\def\blfootnote{\gdef\@thefnmark{}\@footnotetext}
\makeatother

%\author{Markus Iser, Felix Kutzner, Carsten Sinz}

\begin{document}

\lstset{%
    basicstyle=\ttfamily\scriptsize, %\small,
    captionpos=b,
    %backgroundcolor=\color{gray!10},%
    escapeinside={@}{@},
    aboveskip=7pt,
    abovecaptionskip=3pt,
    belowskip=3pt,
    boxpos=c,
    showspaces=false,
    showtabs=false,
    tabsize=4,
    breaklines=true,
    columns=flexible,
    mathescape,
    language=C,
    keywordstyle=\color{teal}\bfseries,
    stringstyle=\color{olive},
    commentstyle=\color{gray}\itshape
}

\title{The LLBMC Family of Benchmarks}

\author{\IEEEauthorblockN{Markus Iser\IEEEauthorrefmark{1},
Felix Kutzner\IEEEauthorrefmark{2} and Carsten Sinz\IEEEauthorrefmark{3} }
\IEEEauthorblockA{Institute for Theoretical Computer Science,
Karlsruhe Institute of Technology\\
Karlsruhe, Germany\\
Email: \IEEEauthorrefmark{1}markus.iser@kit.edu,
\IEEEauthorrefmark{2}felix.kutzner@qpr-technologies.de,
\IEEEauthorrefmark{3}carsten.sinz@kit.edu}}
\maketitle


\begin{abstract}
This family contains benchmarks from software bounded model checking
generated by the tool LLBMC\footnote{\url{http://llbmc.org}}.
\end{abstract}

\section{Introduction}
LLBMC (the low-level bounded model checker) is a static software analysis
tool for finding bugs in C (and, to some extent, in C++) programs.
It is mainly intended for checking low-level system code and is based on the
technique of Bounded Model Checking.

The files in this benchmark have been generated from small sample programs
and a more realistic embedded C code device driver by extracting the SAT formulas
from proof attempts. In LLBMC, checks on programs are converted to SMT formulas
in the logic of bit-vectors and arrays (QF\_ABV), which are in turn converted to SAT.

The benchmarks have been generated using the SMT solver
STP\footnote{\url{https://github.com/stp/stp}}.

\textsc{QPR-Verify}\footnote{\url{http://qpr-technologies.de}} is an extension
of LLBMC, which is intended for commercial use,
and has been used to generate the SAT instances ``BMP280 Driver''.

\section{Benchmarks}

\subsection{BMP280 Driver}

This benchmark is based on the Bosch Sensortec MEMS pressure sensor
driver \footnote{\url{https://github.com/BoschSensortec/BMP280_driver}},
consisting of two C files (\texttt{bmp280.c}, version V2.0.5, and
\texttt{bmp280\_support.c}, version V1.0.6)
with 1963 lines of code combined.

Each function of the device driver is considered as an entry point for
checking for violations of run-time properties such as integer overflows, or
array index out-of-bounds.

The benchmark contains nine files corresponding to nine functions of the device
drivers that have been checked by \textsc{QPR-Verify} in that way.

\subsection{Array Average}

The Array-Average subcategory contains instances generated by checking 
equivalence of two functions (\texttt{avg\_l} and \texttt{avg\_i})
computing the average of the first $N$ array elements
with LLBMC, where the parameter $N$ is chosen to be between 2 and 10.

The main function (\texttt{\_\_llbmc\_main}) runs both implementations on
a non-deterministically initialized array and checks that they always return equal
values.

Instances where $N$ is a power of two should be simpler to solve due to simplifications
performed by LLBMC and the SMT solver. All instances are unsatisfiable.

\lstinputlisting{listings/array-avg.c}

\subsection{Modmul}

Modmul is a simple performance test for LLBMC
to check that if $x = z \cdot n$, then $x \equiv 0 \bmod n$ (for machine integers
$x$, $z$, and $n$). The different instances of this subcategory have been generated
by adapting the SMT formula generated by LLBMC manually for different bit-widths $b$ of
the integers ($b \in \{ 8, 10, 12, 14, 16, 32 \}$). Formulas for smaller bit-width should
be easier to solve, all instances are unsatisfiable.

\lstinputlisting{listings/mod-test.c}

\subsection{Division-by-5}

The Division-by-5 subcategory checks the equivalence of
two different ways to compute $x/5$ for a positive, 32-bit machine integer $x$.
The instance is unsatisfiable.

\lstinputlisting{listings/div-by-5.c}

\subsection{Fermat-3}

The file from this subcategory tests a restricted version of Fermat's Last Theorem
for an exponent $n$ of 3, i.e. that there are no solutions of the equation
$x^3 + y^3 = z^3$ for suitable machine integers $x$, $y$, and $z$.
The instance is unsatisfiable.

\lstinputlisting{listings/fermat.c}

\subsection{Magic}

This instance checks an assertion in a C program containing shifts and multiplication.
The origin of the program is unknown, it might be related to some optimized computations
involving remainder in signed division by 100.

\lstinputlisting{listings/magic.c}


%\bibliographystyle{splncs03}
%\bibliographystyle{abbrv}
%\bibliography{}

\end{document}
